
\section {HepPID}

The Particle Data Group\cite{pdg} provides a 
standard numbering scheme\cite{scheme} 
for use by Monte Carlo generators.  Most generators attempt to use these 
numbers, but there are occasional differences in implementation.  
HepPID provides a set of free functions which will translate ID numbers 
to and from the PDG numbering scheme.   These functions are designed to be 
used by HepPDT, HepMC, or any other class library.
The current implementation uses the 2008 numbering scheme.\cite{scheme}

\section { Particle Numbering Scheme }

The PDG numbering scheme is explained in full detail in reference \cite{scheme}.

Quarks, leptons, guage bosons, Higgs, and similar particles are assigned
numbers between 1 and 80.  Numbers 81-100 are for generator specific use.
Any particle with an ID of 100 or less is considered a "fundamental" particle.
These particles are listed in Appendix \ref{elem}.

The PDG numbering algorithm for composite particles uses a signed 7 
digit number for each particle:  $\pm nn_rn_Ln_{q_1}n_{q_2}n_{q_3}n_J$.
$n_{q_{1-3}}$ are quark numbers used to specify the quark content.
The rightmost digit, $n_J$ = 2J + 1, gives the total spin of the composite particle.
The scheme does not cover particles of total spin $J>4$.
The fifth digit, $n_L$, is reserved to distinguish mesons of the
same total ($J$) but different spin ($S$) and orbital ($L$)
angular momentum quantum numbers.
The sixth digit, $n_r$, is used to label mesons radially excited
above the ground state.

Many states appearing in the PDG meson listing do not yet have definite
$q\bar q$ model assignments.  For these states, $n_{q_{2-3}}$
and $n_J$ are assigned according to the state's most likely flavors
and spin.  Within these groups $n_L=0,1,2,\dots$ is used to distinguish
states of increasing mass. These states are flagged with $n=9$.

The numbering scheme does not extend to baryons with $n>0$, $n_r>0$, or $n_L>0$.

Digits $n_{q_2}$ and $n_{q_3}$ are used for mesons, with $n_{q_1}$ = 0.
Digits $n_{q_1}$, $n_{q_2}$, and $n_{q_3}$ are used for baryons.
Digits $n_{q_1}$ and $n_{q_2}$ are used for diquarks, with $n_{q_3}$ = 0. 
(A list of diquark states is in Appendix \ref{elem}.)  
A negative number indicates an antiparticle.  

The states are generally listed in order of increasing mass.  
$K_L^0$ and $K_S^0$ are exceptions.  Their assigned
identification numbers are 130 and 310, respectively.

SUSY particles are indicated with $n=1$ for right-handed particles or $n=2$
for left-handed particles.  Technicolor states have $n=3$.
Excited (composite) quarks and leptons are identified by setting $n=4$.
Other exotic particles have $n n_r=51$.

The new numbering scheme attemts to list all states needed by the
Monte Carlo generators.  Appendix \ref{meson}
contains a full list of meson states and their ID numbers, up through the
top quark states.   
Appendix \ref{baryon} contains a full list of the baryon states.

The baryon $\Xi$ and $\Omega$ states for charmed
and heavier quarks require special consideration.  
Three spin 1/2 states are recognized
for cxy, bxy, etc., where x and y are lighter, non-identical quarks.
The non-primed states are antisymmetric under interchange of the lighter quarks.
and the primed states are symmetric.  The numbering for these states is 
explicitly stated in the new numbering scheme.

In the past, HepPID used an ad-hoc numbering scheme for ions.  
The ad-hoc ion numbers were $1AAAZZZ00n_J$, where
AAA, and ZZZ are the ion's A and Z respectively.

As of PDG 2006\cite{scheme}, nuclear codes are designated by a 
signed 10 digit number: $\pm 10LZZZAAAI$, where 
AAA is the total baryon number and
ZZZ is the total charge.
L is the total number of strange quarks in a hypernucleus.  
I is used to denote excited states.
A hydrogen nucleus ( 1000010010 ) should be identified as a proton ( 2212 )
to avoid confusion.

New numbers identifying magnetic monopoles and black holes have been 
approved for PDG 2010.  

A black hole in models with extra dimensions has code 5000040.  

Magnetic monopoles and dyons are assumed to have one
unit of Dirac monopole charge and a variable integer number
$\pm n_{q_1}n_{q_2}n_{q_3}$ units of electric charge. 
Codes $\pm 411n_{q_1}n_{q_2}n_{q_3}0$ are then used when the magnetic and 
electrical charge sign agree and $\pm 412n_{q_1}n_{q_2}n_{q_3}0$  
when they disagree, with the overall sign of the particle set by the 
magnetic charge.  For now no spin information is provided.

In addition, there is a need to identify "Q-ball" and similar 
very exotic particles which may have large, non-integer charge.
As of HepPDT 3.04.01, these particles are assigned the ad-hoc numbering
$\pm 100XXXY0$, where the charge is XXX.Y.

\subsection { Extending Particle IDs }
\label{pid}

It is expected that any 7 or 10 digit number used as a particle ID will 
adhere to the rules of the Monte Carlo Particle Numbering Scheme 
published by the PDG.\cite{pdg}

In most cases, users can define particles not already in their
particle data table without needing to extend the numbering scheme.
A previously unknown particle can be assigned a valid particle ID by
following the published rules.\cite{scheme}

For convenience, a copy ( montecarlorpp.pdf ) of the Monte Carlo numbering 
scheme document is provided with the installed documentation.

\subsection { Generator Numbering Schemes }

The Isajet particle identification algorithm uses a signed four digit
number: $\pm$MLKJ.  M, L, and K are quarks and J is the spin.  A negative
number indicates the antiparticle, and is meant to associate with the
lightest quark.  For mesons, M = 0, and for diquarks, K = 0.

Pythia, Herwig, EvtGen, and QQ use the PDG algorithm in addition to internal
compressed numbering schemes.
Although the latest implementations of these generators conform closely
to the new numbering scheme, some differences remain.

EvtGen defines a number of pseudo-particles which are just 
conglomerates used by their decay mechanisms.  
Wherever possible, we retain the 
EvtGen numbers for these convenience pseudo-particles.

\subsection { Translating Particle ID's }

The header ParticleIDTranslations.hh defines a number of free functions 
which can be used to translate between generator and standard numbering schemes.
Other functions will be added as need arises.  
Complete code documentation is
on the web at http://lcgapp.cern.ch/project/simu/HepPDT/
or in HepPDT\_reference\_manual.pdf in the installed documentation directory. 

QQ needs extra translation methods for the 
quark pair pseudo-particles since the ID numbers overlap.

\begin{center}
\begin{tabular}{ll}
int & HepPID::translateHerwigtoPDT( const int herwigID); \\
int & HepPID::translateIsajettoPDT( const int isajetID ); \\
int & HepPID::translatePythiatoPDT( const int pythiaID ); \\
int & HepPID::translateEvtGentoPDT( const int evtGenID ); \\
int & HepPID::translatePDGtabletoPDT( const int pdgID); \\
int & HepPID::translateQQtoPDT( const int qqID); \\
int & HepPID::translateQQbar( const int qqID); \\
int & HepPID::translateGeanttoPDT( const int geantID); \\
 \\
int & HepPID::translatePDTtoHerwig( const int pid ); \\
int & HepPID::translatePDTtoIsajet( const int pid ); \\
int & HepPID::translatePDTtoPythia( const int pid ); \\
int & HepPID::translatePDTtoEvtGen( const int pid ); \\
int & HepPID::translatePDTtoPDGtable( const int pid ); \\
int & HepPID::translatePDTtoQQ( const int pid ); \\
int & HepPID::translateInverseQQbar( const int pid ); \\
int & HepPID::translatePDTtoGeant( const int pid ); \\
 \\
void & writeHerwigTranslation( std::ostream \& os ); \\
void & writeIsajetTranslation( std::ostream \& os ); \\
void & writePythiaTranslation( std::ostream \& os ); \\
void & writeEvtGenTranslation( std::ostream \& os ); \\
void & writePDGTranslation( std::ostream \& os ); \\
void & writeQQTranslation( std::ostream \& os ); \\
\end{tabular}
\end{center}

The translation methods use maps which are initalized by the first call 
to that translation.
Because the maps are static, this initialization only happens once.
We use a data table so that compile time is not impacted.

You may also get or check the name of a particle.  
In addition, you may lookup an ID associated with a particle name.  
This will only work if you use the HepPID names. 
Use HepPDT to lookup particle ID's using the names of the 
particles in your ParticleDataTable.

\begin{center}
\begin{tabular}{ll}
std::string &  particleName( const int \& pid ); \\
int         &  particleName( const std::string \& name ); \\
void        &  listParticleNames( std::ostream \& os ); \\
bool        &  validParticleName( const int \& pid ); \\
bool        &  validParticleName( const std::string \& name ); \\
\end{tabular}
\end{center}

\def\etal{{\it et al.}}

\begin{thebibliography}{9}

\bibitem{pdg} 
http://pdg.lbl.gov/ 

\bibitem{scheme}
Particle Data Group: C. Amsler \etal, \emph{Physics Letters} \textbf{B667}, (2008) 1,
\newline
http://pdg.lbl.gov/2008/mcdata/mc\_particle\_id\_contents.html

\bibitem{2006}
Particle Data Group: W.-M. Yao \etal, \emph{J. Phys.} \textbf{G 33}, 314 (2006),
\newline
http://pdg.lbl.gov/2006/mcdata/mc\_particle\_id\_contents.html

\bibitem{2004} 
Particle Data Group: S. Eidelman  \etal, \emph{Physics Letters} \textbf{B592}, (2004) 292, 
\newline
http://pdg.lbl.gov/2004/mcdata/mc\_particle\_id\_contents.html

\end{thebibliography}

\newpage

